
\documentclass[conference]{IEEEtran}
\usepackage{cite}

\usepackage{multirow}
\usepackage[dvips]{epsfig}
\usepackage{graphicx}
\usepackage{caption}
\usepackage{subcaption}
\usepackage{amsmath}
%\usepackage{enumitem}

\newtheorem{step}{\textrm{Step}}
\hyphenation{op-tical net-works semi-conduc-tor}


\begin{document}
%
% paper title
% can use linebreaks \\ within to get better formatting as desired
\title{ASASDASDASDASD}


% author names and affiliations
% use a multiple column layout for up to three different
% affiliations
\author{\IEEEauthorblockN{Seungil Park}
\IEEEauthorblockA{Department of Electrical \& Computer Engineering and \\INMC, Seoul National University, Seoul, Korea\\
Email: spark11@mwnl.snu.ac.kr}
}


\maketitle


\begin{abstract}
Recently, Device-to-Device (D2D) communication have been under discussion with great attention. In particular, D2D discovery, as the first step of D2D communication, is a matter in dispute. In this paper, we introduce the existing works related D2D discovery and propose our scheme to make D2D discovery more efficient in terms of participation time. For that purpose, it requires a little amount of reserved resource, but it is expected that it scarcely affects the performance of all users.
For verification of our work, we analyze our work both in mathematical and simulation-based ways. Through our thorough investigation with a few of resource selection algorithms, we conclude that our proposed scheme provides the clear performance gain in participation time and does not degrade the performance in guaranteed distance between users which share the same resource.

github testtest
\end{abstract}
% IEEEtran.cls defaults to using nonbold math in the Abstract.
% This preserves the distinction between vectors and scalars. However,
% if the conference you are submitting to favors bold math in the abstract,
% then you can use LaTeX's standard command \boldmath at the very start
% of the abstract to achieve this. Many IEEE journals/conferences frown on
% math in the abstract anyway.

% no keywords


\IEEEpeerreviewmaketitle



\section{Introduction}
% no \IEEEPARstart
There have been a lot of interest 




\end{document}
